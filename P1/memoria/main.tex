%%%%%%%%%%%%%%%%%%%%%%%%%%%%%%%%%%%%%%%%%
% Short Sectioned Assignment LaTeX Template Version 1.0 (5/5/12)
% This template has been downloaded from: http://www.LaTeXTemplates.com
% Original author:  Frits Wenneker (http://www.howtotex.com)
% License: CC BY-NC-SA 3.0 (http://creativecommons.org/licenses/by-nc-sa/3.0/)
%%%%%%%%%%%%%%%%%%%%%%%%%%%%%%%%%%%%%%%%%

%----------------------------------------------------------------------------------------
%	PACKAGES AND OTHER DOCUMENT CONFIGURATIONS
%----------------------------------------------------------------------------------------

\documentclass[paper=a4, fontsize=11pt]{scrartcl} % A4 paper and 11pt font size

% ---- Entrada y salida de texto -----

\usepackage[T1]{fontenc} % Use 8-bit encoding that has 256 glyphs
\usepackage[utf8]{inputenc}

% ---- Idioma --------

\usepackage[spanish, es-tabla]{babel} % Selecciona el español para palabras introducidas automáticamente, p.ej. "septiembre" en la fecha y especifica que se use la palabra Tabla en vez de Cuadro

% ---- Otros paquetes ----

\usepackage{amsmath,amsfonts,amsthm} % Math packages
\usepackage{graphics,graphicx, floatrow} %para incluir imágenes y notas en las imágenes
\usepackage{graphics,graphicx, float} %para incluir imágenes y colocarlas
\usepackage{hyperref} % url in references
\usepackage{listings}
\usepackage{color}
\definecolor{grey}{gray}{0.9}

% Para hacer tablas comlejas
\usepackage{multirow}
\usepackage{threeparttable}

\usepackage{fancyhdr} % Custom headers and footers
\pagestyle{fancyplain} % Makes all pages in the document conform to the custom headers and footers
\fancyhead{} % No page header - if you want one, create it in the same way as the footers below
\fancyfoot[L]{} % Empty left footer
\fancyfoot[C]{} % Empty center footer
\fancyfoot[R]{\thepage} % Page numbering for right footer
\renewcommand{\headrulewidth}{0pt} % Remove header underlines
\renewcommand{\footrulewidth}{0pt} % Remove footer underlines
\setlength{\headheight}{13.6pt} % Customize the height of the header

\numberwithin{equation}{section} % Number equations within sections (i.e. 1.1, 1.2, 2.1, 2.2 instead of 1, 2, 3, 4)
\numberwithin{figure}{section} % Number figures within sections (i.e. 1.1, 1.2, 2.1, 2.2 instead of 1, 2, 3, 4)
\numberwithin{table}{section} % Number tables within sections (i.e. 1.1, 1.2, 2.1, 2.2 instead of 1, 2, 3, 4)

\setlength\parindent{0pt} % Removes all indentation from paragraphs - comment this line for an assignment with lots of text

\newcommand{\horrule}[1]{\rule{\linewidth}{#1}} % Create horizontal rule command with 1 argument of height

\usepackage{textcomp}
\usepackage{hyperref}

%----------------------------------------------------------------------------------------
%	DATOS
%----------------------------------------------------------------------------------------

\newcommand{\myName}{Francisco Javier Bolívar Lupiáñez}
\newcommand{\myDegree}{Máster en Ingeniería Informática}
\newcommand{\myFaculty}{E. T. S. de Ingenierías Informática y de Telecomunicación}
\newcommand{\myDepartment}{Ciencias de la Computación e Inteligencia Artificial}
\newcommand{\myUniversity}{\protect{Universidad de Granada}}
\newcommand{\myLocation}{Granada}
\newcommand{\myTime}{\today}
\newcommand{\myTitle}{Práctica 1}
\newcommand{\mySubtitle}{Competición en Kaggle sobre Clasificación Binaria}
\newcommand{\mySubject}{Sistemas Inteligentes para la Gestión de la Empresa}
\newcommand{\myYear}{2016-2017}

%----------------------------------------------------------------------------------------
%	PORTADA
%----------------------------------------------------------------------------------------


\title{	
	\normalfont \normalsize 
	\textsc{\textbf{\mySubject \space (\myYear)} \\ \myDepartment} \\[20pt] % Your university, school and/or department name(s)
	\textsc{\myDegree \\[10pt] \myFaculty \\ \myUniversity} \\[25pt]
	\horrule{0.5pt} \\[0.4cm] % Thin top horizontal rule
	\huge \myTitle: \mySubtitle \\ % The assignment title
	\horrule{2pt} \\[0.5cm] % Thick bottom horizontal rule
	\normalfont \normalsize
}

\author{
	\myName \\ 
	\texttt{fblupi} \\
	\small Posición: X, Puntuación: 0.841
}

\date{\myTime} % Incluye la fecha actual
%----------------------------------------------------------------------------------------
%	INDICE
%----------------------------------------------------------------------------------------

\begin{document}
	
\definecolor{light-gray}{gray}{0.95}
	
\lstset {
	basicstyle=\scriptsize,
	frame=single,
	backgroundcolor=\color{grey}
}

\lstdefinestyle{R}{
	frame=single,
	numbers=left,
	language=R,
	basicstyle=\tiny\ttfamily,
	keywordstyle=\bfseries,
	commentstyle=\itshape,
	identifierstyle=\bfseries,
}
	
\setcounter{page}{0}

\maketitle % Muestra el Título
\thispagestyle{empty}

\newpage %inserta un salto de página

\tableofcontents % para generar el índice de contenidos

%\listoftables
%\listoffigures

\newpage

%----------------------------------------------------------------------------------------
%	DOCUMENTO
%----------------------------------------------------------------------------------------

\section{Exploración de datos}
\label{sec:exploracion-datos}

La primera fase que hay que realizar ante cualquier problema de este tipo es la exploración de datos para poner los cimientos sobre los que realizar el posterior proceso de análisis.
\\ \\
Los datos con los que contamos en el \textit{dataset} son \cite{KaggleTitanicData}:

\begin{itemize}
	\item \texttt{PassengerId}: Id único, no va a ser de utilidad
	\item \texttt{PClass}: Clase en la que viajaba
	\item \texttt{Name}: Nombre y título
	\item \texttt{Sex}: Sexo
	\item \texttt{Age}: Edad
	\item \texttt{SibSp}: Número de hermanos y/o cónyuges
	\item \texttt{Parch}: Número de padres y/o hijos
	\item \texttt{Ticket}: Id del billete
	\item \texttt{Fare}: Tarifa del viaje
	\item \texttt{Cabin}: Cabina donde se alojó
	\item \texttt{Embarked}: Dónde embarcó
\end{itemize}

De aquí podemos concluir que ni \texttt{PassengerId}, ni \texttt{Ticket} nos van a ser útiles pues son únicos. \texttt{Name} podría tener el mismo problema, pero se podría extraer el título para utilizarlo como información que puede resultar útil.
\\ \\
También hay otras variables como \texttt{Cabin} que son poco útiles pues tiene la mayoría de los valores perdidos.

\subsection{La importancia del sexo, la edad y la clase}

Se sabe que el protocolo por aquel entonces era el de mujeres y niños primero por lo que vamos a ver si de verdad se cumplió y nos puede ayudar a predecir quién murió y quién no \cite{TrevorStephensTutorial}.
\\ \\
En primer lugar, vamos a ver el porcentaje de supervivientes:

\begin{lstlisting}[style=R]
prop.table(table(train$Survived))
\end{lstlisting}

\begin{table}[H]
	\centering
	\caption{Porcentaje de supervivientes}
	\label{tab:die-survive}
	\begin{tabular}{|ll|}
		\hline
		Die       & Survive   \\ \hline
		0.6161616 & 0.3838384 \\ \hline
	\end{tabular}
\end{table}

de sexo:

\begin{lstlisting}[style=R]
prop.table(table(train$Sex))
\end{lstlisting}

\begin{table}[H]
	\centering
	\caption{Porcentaje de sexo}
	\label{tab:male-female}
	\begin{tabular}{|ll|}
		\hline
		Male     & Female   \\ \hline
		0.647587 & 0.352413 \\ \hline
	\end{tabular}
\end{table}

y de adultos (considerándose a partir de los 18 años):

\begin{lstlisting}[style=R]
train$Child <- 0
train$Child[train$Age < 18] <- 1
prop.table(table(train$Child))
\end{lstlisting}


\begin{table}[H]
	\centering
	\caption{Porcentaje de edad}
	\label{tab:child-adult}
	\begin{tabular}{|ll|}
		\hline
		Child     & Adult     \\ \hline
		0.1268238 & 0.8731762 \\ \hline
	\end{tabular}
\end{table}

Combinando edad y sexo tenemos el siguiente resultado:

\begin{lstlisting}[style=R]
aggregate(Survived ~ Child + Sex, data=train, FUN=function(x) {sum(x)/length(x)})
\end{lstlisting}

\begin{table}[H]
	\centering
	\caption{Supervivencia según edad y sexo}
	\label{tab:age-sex}
	\begin{tabular}{|ll|l|}
		\hline
		Age   & Sex    & Survived  \\ \hline
		Adult & Female & 0.7528958 \\
		Child & Female & 0.6909091 \\
		Adult & Male   & 0.1657033 \\
		Child & Male   & 0.3965517 \\ \hline
	\end{tabular}
\end{table}

con el que podríamos hacer un primer envío dando por supervivientes a todas las mujeres. Solo con eso se tendría una puntuación de 0.76555.
\\ \\
No obstante, podemos intentar afinar más pues vemos que los hombres que son niños casi llega al 50\% de posibilidades de supervivencia. Agregando otros datos importantes como la clase y la tarifa del viaje obtenemos:

\begin{lstlisting}[style=R]
train$Fare2 <- '30+'
train$Fare2[train$Fare < 30 & train$Fare >= 20] <- '20-30'
train$Fare2[train$Fare < 20 & train$Fare >= 10] <- '10-20'
train$Fare2[train$Fare < 10] <- '<10'
aggregate(Survived ~ Child + Fare2 + Pclass + Sex, data=train, FUN=function(x) {sum(x)/length(x)})
\end{lstlisting}

\begin{table}[H]
	\centering
	\caption{Supervivencia según edad, sexo, clase y tarifa}
	\label{tab:age-sex-fare-pclass}
	\begin{tabular}{|llll|l|}
		\hline
		Age   & Fare        & Pclass & Sex    & Survived   \\ \hline
		Adult & 20-30       & 1      & Female & 0.83333333 \\
		Adult & 30+         & 1      & Female & 0.98750000 \\
		Child & 30+         & 1      & Female & 0.87500000 \\
		Adult & 10-20       & 2      & Female & 0.90625000 \\
		Child & 10-20       & 2      & Female & 1.00000000 \\
		Adult & 20-30       & 2      & Female & 0.88000000 \\
		Child & 20-30       & 2      & Female & 1.00000000 \\
		Adult & 30+         & 2      & Female & 1.00000000 \\
		Child & 30+         & 2      & Female & 1.00000000 \\
		Adult & \textless10 & 3      & Female & 0.56140351 \\
		Child & \textless10 & 3      & Female & 0.85714286 \\
		Adult & 10-20       & 3      & Female & 0.50000000 \\
		Child & 10-20       & 3      & Female & 0.73333333 \\
		Adult & 20-30       & 3      & Female & 0.40000000 \\
		Child & 20-30       & 3      & Female & 0.16666667 \\
		Adult & 30+         & 3      & Female & 0.11111111 \\
		Child & 30+         & 3      & Female & 0.14285714 \\
		Adult & \textless10 & 1      & Male   & 0.00000000 \\
		Adult & 20-30       & 1      & Male   & 0.40000000 \\
		Adult & 30+         & 1      & Male   & 0.35365854 \\
		Child & 30+         & 1      & Male   & 1.00000000 \\
		Adult & \textless10 & 2      & Male   & 0.00000000 \\
		Adult & 10-20       & 2      & Male   & 0.11864407 \\
		Child & 10-20       & 2      & Male   & 0.75000000 \\
		Adult & 20-30       & 2      & Male   & 0.04761905 \\
		Child & 20-30       & 2      & Male   & 0.75000000 \\
		Adult & 30+         & 2      & Male   & 0.00000000 \\
		Child & 30+         & 2      & Male   & 1.00000000 \\
		Adult & \textless10 & 3      & Male   & 0.10931174 \\
		Child & \textless10 & 3      & Male   & 0.15384615 \\
		Adult & 10-20       & 3      & Male   & 0.12903226 \\
		Child & 10-20       & 3      & Male   & 0.71428571 \\
		Adult & 20-30       & 3      & Male   & 0.07142857 \\
		Child & 20-30       & 3      & Male   & 0.20000000 \\
		Adult & 30+         & 3      & Male   & 0.41666667 \\
		Child & 30+         & 3      & Male   & 0.07692308 \\ \hline
	\end{tabular}
\end{table}

donde podríamos afinar diciendo que si las mujeres son de la clase 3 y su tarifa es mayor a 20 murieron y los niños con tarifa mayor a 30 de clase 1 o 2 sobrevivieron. Con este pequeño árbol de decisión hecho manualmente explorando los datos obtenemos una puntuación mejor que la anterior de un total de 0.77990.

\section{Preprocesamiento de datos}

En esta sección trataremos valores perdidos así como extraer nuevo conocimiento de distintas variables \cite{TrevorStephensTutorial}.

\subsection{Nombres y familias}

Como se comentó en la Sección \ref{sec:exploracion-datos}, los nombres podrían parecer únicos, pero se puede extraer información para clasificarlos por títulos y familias pues títulos más importantes podrían haber tenido prioridad para salvarse y las familias, al ir lo más probable juntas, tendrían posibilidad de sobrevivir o morir todos.
\\ \\
Esto se ha hecho tanto con los \textit{dataset} de \textit{train} como de \textit{test}:

\begin{lstlisting}[style=R]
combi <- rbind(train, test)
\end{lstlisting}

\subsubsection{Títulos}

En primer lugar vamos a extraer el título de cada una de las entradas y agregarlo como un atributo más:

\begin{lstlisting}[style=R]
combi$Name <- as.character(combi$Name)
combi$Title <- sapply(combi$Name, FUN=function(x) {strsplit(x, split='[,.]')[[1]][2]})
combi$Title <- sub(' ', '', combi$Title)
\end{lstlisting}

Con esto obtenemos:

\begin{itemize}
	\item \textit{Capt}: 1
	\item \textit{Col}: 4
	\item \textit{Don}: 1
	\item \textit{Dona}: 1
	\item \textit{Dr}: 8
	\item \textit{Jonkheer}: 1
	\item \textit{Lady}: 1
	\item \textit{Major}: 2
	\item \textit{Master}: 61
	\item \textit{Miss}: 260
	\item \textit{Mlle}: 2
	\item \textit{Mme}: 1
	\item \textit{Mr}: 757
	\item \textit{Mrs}: 197
	\item \textit{Ms}: 2
	\item \textit{Rev}: 8
	\item \textit{Sir}: 1
	\item \textit{the Countess}: 1
\end{itemize}

Agregando los grupos pequeños que significan los mismo:

\begin{lstlisting}[style=R]
combi$Title[combi$Title %in% c('Mme', 'Mlle')] <- 'Mlle'
combi$Title[combi$Title %in% c('Capt', 'Don', 'Major', 'Sir')] <- 'Sir'
combi$Title[combi$Title %in% c('Dona', 'Lady', 'the Countess', 'Jonkheer')] <- 'Lady'
combi$Title <- factor(combi$Title)
\end{lstlisting}

obtenemos:

\begin{itemize}
	\item \textit{Col}: 4
	\item \textit{Dr}: 8
	\item \textit{Lady}: 4
	\item \textit{Master}: 61
	\item \textit{Miss}: 260
	\item \textit{Mlle}: 3
	\item \textit{Mr}: 757
	\item \textit{Mrs}: 197
	\item \textit{Ms}: 2
	\item \textit{Rev}: 8
	\item \textit{Sir}: 5
\end{itemize}

\subsubsection{Familias}

Las familias grandes podrían tener más posibilidad de morir buscando a cada uno de los miembros para salvarse todos.
\\ \\
Lo primero que hay que hacer es combinar los campos \texttt{SibSp} y \texttt{Parch} que como vimos en la Sección \ref{sec:exploracion-datos} eran los que determinaban hermanos y/o cónyuges y padres y/o hijos:

\begin{lstlisting}[style=R]
combi$FamilySize <- combi$SibSp + combi$Parch + 1
\end{lstlisting}

Ahora extraeremos el apellido:

\begin{lstlisting}[style=R]
combi$Surname <- sapply(combi$Name, FUN=function(x) {strsplit(x, split='[,.]')[[1]][1]})
\end{lstlisting}

Ahora agregaremos ambos valores y daremos valor \textit{Small} a los que tienen 3 familiares o menos:

\begin{lstlisting}[style=R]
combi$FamilyID <- paste(as.character(combi$FamilySize), combi$Surname, sep="")
combi$FamilyID[combi$FamilySize <= 3] <- 'Small'
\end{lstlisting}

No obstante vemos que a veces no coincide el id con el número de miembros en esto. Esto puede ser porque haya familiares con distinto id por no estar casados o no haber cambiado el apellido de soltería. Para lidiar con esto se les dará valor \textit{Small}:

\begin{lstlisting}[style=R]
famIDs <- data.frame(table(combi$FamilyID))
famIDs <- famIDs[famIDs$Freq <= 3,]
combi$FamilyID[combi$FamilyID %in% famIDs$Var1] <- 'Small'
combi$FamilyID <- factor(combi$FamilyID)
\end{lstlisting}

\subsection{Edad}

El campo de edad tiene muchísimos valores perdidos (263) entre los datos de \textit{train} y de \textit{test}.

\begin{lstlisting}[style=R]
summary(combi$Age)
\end{lstlisting}

Para resolver esto, se va a realizar una predicción del valor de estas variables usando el árbol de decisión de la librería \texttt{rpart} con el método \texttt{anova} pues queremos predecir una variable contínua y no categórica.

\begin{lstlisting}[style=R]
Agefit <- rpart(Age ~ Pclass + Sex + SibSp + Parch + Fare + Embarked + Title + FamilySize,
                data=combi[!is.na(combi$Age),], 
                method="anova")
combi$Age[is.na(combi$Age)] <- predict(Agefit, combi[is.na(combi$Age),])
\end{lstlisting}

\subsection{Embarque}

El embarque tiene dos valores perdidos:

\begin{lstlisting}[style=R]
summary(combi$Embarked)
\end{lstlisting}

Como la mayoría salió desde Southampton, vamos a darle este valor:

\begin{lstlisting}[style=R]
which(combi$Embarked == '') # 62 830
combi$Embarked[c(62,830)] = "S"
combi$Embarked <- factor(combi$Embarked)
\end{lstlisting}

\subsection{Tarifa}

La tarifa tiene un valor perdido:

\begin{lstlisting}[style=R]
summary(combi$Fare)
\end{lstlisting}

Por lo que le daremos la mediana:

\begin{lstlisting}[style=R]
which(is.na(combi$Fare)) # 1044
combi$Fare[1044] <- median(combi$Fare, na.rm=TRUE)
\end{lstlisting}

%----------------------------------------------------------------------------------------
%	REFERENCIAS
%----------------------------------------------------------------------------------------

\newpage

\bibliography{referencias} %archivo referencias.bib que contiene las entradas 
\bibliographystyle{plain} % hay varias formas de citar

\end{document}